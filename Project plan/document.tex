\documentclass[11pt,a4paper]{article}
\usepackage[a4paper]{geometry}
\usepackage{graphicx}
\usepackage{xcolor}
\usepackage{float}

\usepackage{amsmath, amscd, amsthm, amssymb, mathrsfs,amsfonts}
\usepackage{subfigure}
\usepackage[font=small,labelfont=bf]{caption}
\usepackage{listings}
\usepackage{color} %red, green, blue, yellow, cyan, magenta, black, white
\definecolor{mygreen}{RGB}{28,172,0} % color values Red, Green, Blue
\definecolor{mylilas}{RGB}{170,55,241}

\lstset{language=Java,%
	basicstyle=\small \ttfamily \bfseries,%basicstyle=\color{red},
	breaklines=true,%
	morekeywords={matlab2tikz},
	keywordstyle=\color{blue},%
	morekeywords=[2]{1}, keywordstyle=[2]{\color{black}},
	identifierstyle=\color{black},%
	stringstyle=\color{mylilas},
	commentstyle=\color{mygreen},%
	showstringspaces=false,%without this there will be a symbol in the places where there is a space
	numbers=left,%
	numberstyle={\tiny \color{black}},% size of the numbers
	numbersep=9pt, % this defines how far the numbers are from the text
	emph=[1]{for,end,break},emphstyle=[1]\color{red}, %some words to emphasise
	%emph=[2]{word1,word2}, emphstyle=[2]{style}, 
	captionpos=t, 
	frame=single,
	title=\lstname
}

\title{Design of Multi-Agent Systems: Riot Control Strategies}
\author{Maikel Withagen (s1867733) \and Steven Bosch (s1861948) \and Robin Kramer}
\date{\today}

\begin{document}
	\maketitle
	
	\section{}	
		Problem: How Effective are Different Strategies for Controlling Aggressive Group Behavior
		State-of-the-Art: Goh et al., Modeling Civil Violence: An Evolutionary Multi-Agent, Game-Theoretic Approach.
		New Idea: Explore the influence of different predefined agent behaviors on the success to control group aggression, for instance, in the city. 
		Results: A simulation with agent agents having a common goal, during  the control of group aggression. Success is defined by several factors such as the amount of civilian casualties and the amount of exterminated hostiles.
		Relevance: Simulation results can provide explanation for human behavior under different circumstances. Moreover, it might provide an answer to which strategy is most effective in controlling riots. 
		
		
		Uitwerking Simulatie:
		De simulatie bestaat uit drie groepen. Er is een groep agent agenten, en een groep relschoppers en een groep burgers. Deze relschopper hebben altijd hetzelfde gedrag. Hun doel is om zoveel mogelijk agenten (en burgers) uit te schakelen. De acties die de relschoppers nemen hangen af van de hoeveelheid mede-relschoppers en de hoeveelheid agenten. 
		Voor de agenten implementeren we vier verschillende acties welke tijdens een simulatie genomen kunnen worden. Er zijn twee doelen: Het uitschakelen van de tegenstanders en het tot veiligheid brengen van burgers. Deze twee doelen kunnen in verschillende gradaties tegelijk voorkomen.  
		Er zijn meerdere scenario’s waarin we de aantal en de spreiding van de relschoppers veranderen. Wij verwachten dat het gedrag van de relschoppers iets anders zal verlopen. Dit zou ook betekenen dat verschillende strategieën effectief kunnen zijn. Elk scenario zal met elke strategie gecombineerd worden. 

\end{document}