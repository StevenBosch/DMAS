\documentclass[11pt,a4paper]{article}
\usepackage[a4paper]{geometry}
\usepackage{graphicx}
\usepackage{xcolor}
\usepackage{float}

\usepackage{amsmath, amscd, amsthm, amssymb, mathrsfs,amsfonts}
\usepackage{subfigure}
\usepackage[font=small,labelfont=bf]{caption}
\usepackage{listings}
\usepackage{color} %red, green, blue, yellow, cyan, magenta, black, white
\definecolor{mygreen}{RGB}{28,172,0} % color values Red, Green, Blue
\definecolor{mylilas}{RGB}{170,55,241}

\usepackage[round]{natbib}   % omit 'round' option if you prefer square brackets
\bibliographystyle{plainnat}
\usepackage{enumitem}

\lstset{language=Java,%
	basicstyle=\small \ttfamily \bfseries,%basicstyle=\color{red},
	breaklines=true,%
	morekeywords={matlab2tikz},
	keywordstyle=\color{blue},%
	morekeywords=[2]{1}, keywordstyle=[2]{\color{black}},
	identifierstyle=\color{black},%
	stringstyle=\color{mylilas},
	commentstyle=\color{mygreen},%
	showstringspaces=false,%without this there will be a symbol in the places where there is a space
	numbers=left,%
	numberstyle={\tiny \color{black}},% size of the numbers
	numbersep=9pt, % this defines how far the numbers are from the text
	emph=[1]{for,end,break},emphstyle=[1]\color{red}, %some words to emphasise
	%emph=[2]{word1,word2}, emphstyle=[2]{style}, 
	captionpos=t, 
	frame=single,
	title=\lstname
}

\title{Design of Multi-Agent Systems: Riot Control Strategies}
\author{Maikel Withagen (s1867733) \and Steven Bosch (s1861948) \and Robin Kramer (1970755)}
\date{\today}

\begin{document}
	\maketitle
	
	\section{Problem}
		What are the effects of altering individual behaviour caused by experiences on group effectiveness in suppressing civil violence?
		
	\section{State of the art}
		The article we use as state of the art is ``Modeling Civil Violence: An Evolutionary Multi-Agent, Game-Theoretic Approach" by Goh et al.
		
		In their article \citet{Goh} focus on the design and development of a spatial Multi-Agent Social Network (EMAS) model for examining the emergent macroscopic-behavioural dynamics of civil violence, as a result of microscopic local movement and game-theoretic interactions between multiple goal-oriented agents in various scenarios. Their evolutionary approach is based solely on performance of agents in terms of wins and losses (determined by iterated prisoner's dilemma results). Agents look at the performance of other agents and try to imitate top performers by copying their behaviour. 
		
	\section{New idea}
		Our new idea involves exploring the influence of past experiences of individual agents on their behaviour and consequently on their success in suppressing civil violence. So in contrast to Goh et al.'s approach, agents adjust their behaviour based on their own experience instead of on success of other agents. 
		
	\section{Method}
		The simulation consists of several cells in a grid, in which three groups of agents are simulated: peacekeepers, hostiles and civilians. As soon as aggression is decreased to an acceptable degree in all cells, the riot is considered contained. Peacekeepers learn from their experiences during the riot, which influences their behaviour during following riots. In each step in the simulation peacekeepers clash with hostiles in the same cell and peacekeepers can choose between multiple actions, such as aggressive fire, cautious fire etc. What action to perform is dependent on a number of scores:
		
		\begin{itemize}
			\item Danger score, which is based upon the number of hostiles and their aggressiveness.
			\item Risk score, which is based upon the proportion civilans/hostiles, the chosen agent action and the proportion agents/hostiles)
			\item Intention weights of a peacekeeper. These weights determine the probability of the possible actions associated with different levels of aggressiveness.
		\end{itemize}
		
		We keep the simulation stochastic, in the sense that an agent's action is never fully determined beforehand. All of factors mentioned above influence the probability of choosing certain actions, just as in the real world where people's behaviours differ from one another and cannot be predicted with certainty.
	
		The goal of the peacekeepers is based on two factors: to minimize civilian casualties and to disperse the hostiles. Their own goals can change over time based on their own experiences. 
		
	\section{Results}
		Using a microscopic implementation we look at the results on a macroscopic level.
		First, we look at the degree of adaptivity of the peacekeepers' behaviour in relation to the success of suppressing civil violence in the short and long term. 
		Another interesting result will be whether a new group or an experienced group of agents is better at keeping civil violence under control.
		
	\section{Relevance}
		Simulation results can provide explanations for human behaviour under different circumstances. Moreover, it might provide insight into the effects of individual experiences on group effectiveness in suppressing riots.
		
	
	\bibliography{mybib}

\end{document}