\section{Results}
The first three simulations were run with learning rate values of $0.8$, $0.5$, and $0.2$, corresponding to the inexperienced, mixed-experienced and experienced cops. Following, the simulation with a decreasing learning rate, starting at $1.0$ was run. The mean success value can be found in \autoref{tab:ResultsCur}. 


\begin{table}
\begin{center}
\begin{tabular}{l r  r}
$\lambda$ & mean \# epochs & Success \\
\hline
0.8 & 55.6 & 0.218 \\
0.5 & 46.8 & 0.230 \\
0.2 & 58.2 & 0.210 \\
<1.0 & 39.2 & 0.235 \\
\hline
\end{tabular}
\caption{The amount of epochs necessary to finish and the success of the simulation for the cop-teams with different levels of experience. }
\label{tab:ResultsCur}
\end{center}
\end{table}

A Repeated Measure ANOVA showed that there was no main effect of the different simulation on both the success (F(5,25) = 387; p = 0.853) or the amount of epochs (F(5,24) = 1.396; p = 0.261). Post-hoc pairwise comparison, using Bonferoni Correction, confirmed this, showing only non-significant differences between the different simulations.

 