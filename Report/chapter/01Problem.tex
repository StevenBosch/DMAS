\section{Introduction}
\subsection{Problem}
The civil war in Bosnia and Herzegovina from 1992 to 1995, the Ferguson unrest in 2014, and more recently the civil war in the Ukrain; Civil violence has been an issue in many different times and places. There is no single cause for these riots and wars. Some were the result of cultural differences, some of the feeling of being treated unjustly, and others came about due to political reasons. ``Each war is as different as the society producing it" \citep*{sambanis2001ethnic}, and getting more insight into the development and handling of these riots is of utmost importance. 

\subsection{State of the Art}
Much research has been done to understand how these riots emerge, including simulations of riots using a game theoretic approach \citep*{myerson1991game} and social networks \citep*{gulden2002spatial} among others. These investigations have shown that the behavior at a macroscopic level, that is, the behavior of an entire group, is the result of behavior on the microscopic level, that is, the individual agents of that group. In a more recent study, Goh and colleagues \citep*{goh2006modeling} also studied how macroscopic behavior emerged using a game theoretic approach in a simulation. The most important issue the authors addressed was how different events affected the individual tendency to riot and how this in turn affected macroscopic behavior and situations. 

Goh et al. included many different interactions in their simulation, such as a probability that civilians turned to active protesters, how jail time affected rehabilitation of arrested protesters, and how the amount and types of people affected individual decisions. The focus was thus mainly on the civilian and the protesters. What the authors failed to focus on, however, was how experience affected the cops' behavior during riots. It is believed that many previous experiences ensure a more stable performance \citep{anderson2007mind,nason2005soar}, meaning that people switch less often between different strategies for the same goal. In other words, more experienced humans are less affected by new experiences compared to less experienced people, which can roughly be translated to the saying "you can't learn old dogs new tricks." 

\subsection{New Idea}
Knowing how experience influences group behavior may be of vital importance for the success of a group performance. For one individual, it is very well possible that the `correct' course of action may result in a negative outcome in a particular situation, because the world is not deterministic. The experienced person will be resilient to such an `accidental' negative outcome, whereas a less experienced person may be affected more severely by that negative outcome. The latter person may therefore change to a less favourable course of action in a future occurrence of said situation, which may result in frailty of the group's dynamics. It has already been shown in an organizational context that individual experience may determine the success of development of the organization \citep*{reagans2005individual}. Individual experience may thus have a great impact on group performance. As of yet this has not received much attention in a civil violence context however. 

This paper focusses on how experience influences group behavior. By using a simulation of a multi-agent system, we were able to compare how experienced and inexperienced cop agents perform against a group of aggressive hostile agents. A reinforcement learning strategy was used to let the individual agents learn the optimal strategy, which has been shown to be a robust and natural way of teaching agents \citep{claus1998dynamics}. In the following sections, the simulation and data acquisition will be described in more detail. Thereafter the results of the different simulation will be presented. The paper concludes with a discussion of the implications of the results, and some critical notes to this research. 