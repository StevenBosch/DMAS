\chapter{Problem}
The civil war in Bosnia and Herzegovina from 1992 to 1995, the Ferguson unrest in 2014, and more recently the civil war in the Ukrain; Civil violence has been an issue in many countries for many years. There is no single cause for these riots and wars. Some were the result of cultural differences, some because of the feeling of being treated injustly, and others came about due to political reasons. ``Each war is as different as the society producing it", and understanding the reasons for these riots to come about is of upmost importance (Ref5Goh), but also how the authorities respond to these situations should be the focus for investigation. 

Much research has been done to understand how these riots emerge, including simulations of riots using a game theoretic approach (Ref8Goh) and social networks (Ref10Goh) among others. These investigations have shown that the behavior at a macroscopic level, that is the behavior of an entire group, is the result of the microscopic level, that is the behavior of the individual. In a more recent study, Goh and colleagues (Goh, Quek, Tan \& Abbass, 2006) also studied how macroscopic behavior emerged using a game theoretic approach in a simulation. The most important issue the authors adressed was how experiences changed the behavior of the individual and how this learning affected macroscopic behavior. 

Goh included many different interactions in his simulation, such as a probability that civilians turned to active protesters, how jail time affected rehabilitation of arrested protesters, and how the amount and types of people affected individual decisions. The focus was thus mainly on the civilian and the protesters. What the authors failed to focus on was how experience affected the learning rate. In many re\"inforcement learning algorithms, such as Q-learning, the learning rate may decrease over time, allowing utilities of particular actions to converge (Watkins \& Dayan, 1992: Q-Learning, technical notes), and many argue that this decrease in learning rate also occurs in humans (REF, REF). In other words, more experienced humans are less affected by new experiences compared to less experiened people. It is very well possible that the `correct' choice may result in a negative outcome in a particular situation. The experienced person will be resilient to such an `accidental' outcome, whereas a less experienced person may be affected more severely by that negative outcome. The latter person may therefore change to a less favourable course of action in a future occurence of said situation. This may result in frailty of the group's dynamics, ultimately leading to a drop in performance or even a loss. On the other hand, if a less experienced person had some accidental successes with an `incorrect' action, it is easier to learn that `correct' actions are more profitable. Individual experience may thus have a great impact on group performance, but so far, this has not received much attention. 

The current paper focusses on how experience will influence cop behavior. By using a simulation, we were able to compare how experienced cop agents, not experienced cop agents and a mix thereof perform against a group of trained hostile agents. In the followings sections, the simulation and, in turn, data acquisition will be descibed in more detail. Following the results of the different simulation will be presented. The paper concludes with a discussion of the implications of the results, and some shortcomings to this paper. 