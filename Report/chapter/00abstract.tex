\begin{abstract}
	\noindent Much research has been done on how civil violence can emerge,
	but less focus has been on how to control civil violence, 
	e.g.\ what factors affect cop performance while controlling civil riots.
	In this paper the effect of learning from individual experiences of cops on their overall group performance is investigated using a multi-agent simulation.
	Cops that have a lot of experience tend to stick to known and (previously) successful behavior (low learning rate),
	whereas new cops, having less experience,
	are more likely to alter their behavior after unexpected unfavourable outcomes (high learning rate),
	which may affect group dynamics.
	Some significant differences were found between learning rates; these were marginal though.
	A large effect was seen between different biases for strategies.
	An aggressive approach resulted in less casualties at the cost of mitigation speed,
	whereas a more peaceful approach increased the speed of mitigating civil violence at the cost of many cop and civilian casualties.
	Though the current simulation was rather simplistic,
	the results do show that multi-agent modelling is a useful tool for understanding group dynamics,
	and may provide us with the possibility to test and compare civil violence mitigations strategies.
\end{abstract}
