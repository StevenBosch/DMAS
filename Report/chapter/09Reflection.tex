\chapter{Discussion}
In this paper a multi-agent system approach was taken to identify the effects of experience on learning during a group task, that is, riot control. Several simulation were run using different learning rate - this way simulating the age or amount of experience of the agents - after which the success of the riot control was determined. Results showed no significant differences between the different simulation, suggesting that experience is not a factor that affects learning and group performance.

Some shortcomings to this research method do deserve mentioning. First of all, only few simulations were run. Of the four different scenarios, only five simulations were run. Only few results could be obtained, which does not allow for much comparison and interpretations. Having more simulations might improve the statistical significance of the results. %% This will be done differently in the final version of the report. 

Another issue encountered in this multi-agent system was the fact that the learned behavior was not transferred to new simulations. Because of the brevity of the individual simulations, agents were not capable of learning a lot, which might have explained the lack of differences in the results as well. Making the simulations longer and transferring the learned behavior to new simulations, might improve the agents' capability to learn and differences to arise. 

The implication of the results found must be taken with a grain of salt. The simulation is a greatly oversimplified representation of reality. The hostiles only have one goal, which is to kill, whereas the agents only decision is between saving and shooting. Moreover, the agent will only go away if every hostile is taken care of in his area. Also, civilians just disappeared from the city when they were saved. This is obviously not what really happens. Goh et al. \citep{goh2006modeling} had improved the realism of his simulation by including many factors that may happen during a civil war. A combination of their simulation with the current proposed model may allow some transfer from the model to reality, and may therefore allow us to learn from the results. 

\section{Conclusion}
The effects of learning rate of individual agents on group performance during civil war control was researched using a multi-agent system approach. Due to some shortcomings of the model that was used and the lack of results no actual lessons could be learned though. In future studies, many improvements should be made to the model before we can transfer these results to reality. 